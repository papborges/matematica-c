\section*{Apresentação}

O problema da significativa reprovação em Cálculo Diferencial e Integral levou
as universidades brasileiras a ofertar uma disciplina introdutória (Matemática C, na
Universidade Federal da Fronteira Sul, UFFS), com o objetivo de revisar e fundamentar
o estudo de tópicos de Matemática ensinados na Educação Básica. A carência de livros
didáticos com uma abordagem conceitual, na qual o significado, a notação matemática e
as aplicações dos conceitos fossem apresentadas em linguagem acessível a
universitários iniciantes na área das ciências exatas, levou à criação do livro
Matemática: funções elementares.

A primeira versão teve a forma de apostila, composta por 14 capítulos gravados
em arquivos independentes para viabilizar o envio aos alunos. A aplicação em várias
turmas de Matemática C viabilizou o aperfeiçoamento do texto, correção de erros e
inclusão de mais atividades e exercícios. Nessas práticas, também foram observados
problemas de manuseio, desconfiguração e gerenciamento de atualizações. Por outro
lado, observou-se também, que os alunos resistiam à leitura de livros didáticos físicos e
respondiam bem às iniciativas de ensino com tecnologias de comunicação e materiais
dinâmicos, além de participarem ativamente no melhoramento da apostila. Com esses
aprendizados e visando resolver os problemas de edição, atualização e distribuição da
apostila, foi criado o Grupo de Estudos: Material de Ensino de Matemática Elementar
em Formato de Texto Estruturado, com a participação de professores e alunos dos
Cursos de Matemática-Licenciatura e Ciência da Computação da UFFS.

A alternativa técnica escolhida foi a reedição da apostila na linguagem \LaTeX
como texto estruturado, disponibilizado na plataforma GitHub (\href{https://github.com/papborges/matematica-c}{https://github.com/papborges/matematica-c}). Com
isso, foram resolvidos os problemas de manuseio, distribuição e revisão, uma vez que
alunos e professores podem acessar o texto completo livremente, sem ônus, propor
modificações e, de maneira colaborativa, contribuir para o aperfeiçoamento do material
de ensino. Esse formato possibilitou também a inserção de links que viabilizam o
deslocamento rápido de um ponto a outro do livro - conectando exercícios com suas
respostas, conceitos com informações adicionais e curiosidades - tornando a leitura mais
dinâmica e fluente.

Alunos e professores, sintam-se convidados a apreciar e contribuir para o
aperfeiçoamento desse livro.

Equipe do Grupo de Estudos:

\begin{itemize}
    \item Material de Ensino de Matemática Elementar em Formato de Texto Estruturado
\end{itemize}
